\chapter{Introduction}

\section{Motivation}

Graphics processing units (GPUs) have gradually increased in computational power from small, 
job-specific boards to programmable powerhouses. Compared to more common central processing units (CPUs), 
GPUs have a higher aggregate memory bandwidth, much higher floating-point operations per second 
(FLOPs), and lower energy consumption per FLOP \cite{warp2015}.

As we approach exascale computing, many institutions are investing in GPU hardware to help achieve these
peak speeds. In the United States, Oak Ridge National Laboratory (ORNL) and Lawrence Livermore National
Laboratory (LLNL) are both acquiring high-performance computing systems that include NVIDIA GPU 
architectures \cite{ornl, llnl}. The computer system to be built at the Oak Ridge Leadership Computing
Facility will be named ``Summit" and will consist of nodes with multiple IBM POWER9 CPUs and NVIDIA Volta
accelerators. The architecture will use a coherent memory space that includes high bandwidth memory on the
GPUs and a high speed NVLink interconnect between the POWER9 CPU and Volta GPUs \cite{ornl}. ``Sierra",
the next advanced technology high-performance computing system to be introduced into LLNL's High
Performance Computing Complex, will also incorporate IBM POWER and NVIDIA Volta processors \cite{llnl}.
Additionally, it is estimated that Sierra will be about five times more power efficient than its
predecessor, Sequoia, while providing four to six times the sustained performance and five to seven times
the workload performance.

Supercomputing systems that use GPUs to accelerate computation are competitive in terms of speed when 
compared to other types of supercomputers. The TOP500 project uses a benchmark of ability to solve
linear systems of equations in dense matrix form to rank supercomputers. The most recent TOP500 list was
published in November 2014 \cite{top500} and demonstrated that supercomputers using GPU acceleration can
achieve peak speeds. The No.\ 2 and No.\ 6 systems both use NVIDIA GPUs to accelerate computation. 
Overall, fifty of the top 500 systems use NVIDIA chips for acceleration \cite{top500}.

However, transitioning to using these GPU-accelerated supercomputers will require substantial time
and effort. For many algorithms, CPU-optimized parallel algorithms are not directly portable to GPU 
architectures. In the field of nuclear engineering research and simulation specifically, neutral particle
transport codes must be rewritten to execute efficiently on GPUs \cite{warp2015}. In particular, the field
of nuclear engineering uses particle transport codes for reactor analysis, criticality safety, nuclear 
security and nonproliferation, detection, and shielding calculations.

Currently, there are not any general purpose Monte Carlo codes that execute on GPUs.
Two prominent codes used for neutron transport calculations in nuclear reactor analysis are the Monte
Carlo N-Particle (MCNP) code and the Serpent code. Both pieces of software use Monte Carlo methods to 
track neutron histories and perform criticality calculations on CPU-based computer architectures 
\cite{mcnp, serp_man}. At this time, however, there are no plans to adapt the MCNP code for use on GPUs
\cite{mcnp_nogpu}, providing incentive for nuclear engineering researchers to begin developing their own
Monte Carlo neutron transport software for efficient use on GPUs.

\section{WARP}

WARP, which can stand for ``Weaving All the Random Particles", is a three-dimensional (3D), continuous 
energy, Monte Carlo neutron transport code originally developed by Dr. Ryan M. Bergmann at UC Berkeley to 
efficiently execute on a CPU/GPU platform \cite{warp2015}. WARP is able to calculate multiplication 
factors, flux tallies, and fission source distributions for time-independent problems and can run in both 
criticality or fixed-source modes. WARP currently transports neutrons in unrestricted arrangements of 
spheres, cylinders, parallelpipeds, and hexagonal prisms \cite{warp2015} and is able to entertain both 
vacuum and reflecting (specular) boundary conditions.

What sets WARP apart from previous, somewhat similar endeavors is its breadth of scope and novel 
adaptation of the event-based Monte Carlo algorithm. Previous codes have been limited to restricted 
nuclear data or simplified geometry models, where WARP instead loads standard data files and uses a 
flexible, scalable, optimized geometry representation \cite{warp2015}. WARP uses a suite of 
highly-parallelized algorithms and employs a modified version of the original event-based algorithm that 
is better suited to GPU execution. For full detail, please see the paper by Bergmann and Vuji{\'c} (2015) 
\cite{warp2015}.
