\chapter{Implementation}

\section{Delta-tracking in WARP}

The implementation of delta-tracking in WARP is somewhat similar to the implementation in RACER and 
Serpent in that it uses a combination of ray tracing and delta-tracking. 
To understand what is happening in WARP, what follows is an explanation of the general delta-tracking 
algorithm implemented, a comparison to ray tracing, discussion of the specific implementation in WARP, and
finally some comments about the performance potential of using delta-tracking on GPUs for the first time.

In WARP, the delta tracking algorithm executes these steps:

\begin{itemize}
	\item{get current neutron positions using OptiX trace;}
	\item{calculate majorant cross section, sample path lengths, update neutron coordinates;}
	\item{update neutron positions and material information using OptiX trace;}
	\item{calculate material total cross section and perform rejection sampling;}
	\item{determine neutron interaction.}
\end{itemize}

\noindent For direct comparison, tracking particles via ray tracing only in WARP is executed as:

\begin{itemize}
	\item{get current neutron positions using OptiX trace;}
	\item{calculate macroscopic total cross section and sample path length;}
	\item{determine with which isotope neutron interacts;}
	\item{move neutron to collision site or cell boundary, whichever is closer;}
	\item{determine neutron interaction.}
\end{itemize}

\noindent From there, with either tracking scheme, neutrons are grouped based on the interactions that 
they are about to undergo and the vectors are subsequently processed.

\section{Implementation}

In WARP and other Monte Carlo neutron transport codes, particle positions are updated as:

\begin{gather}
\label{eq:move}
x \mathrel{+}= \ell\hat{x}, \nonumber\\
y \mathrel{+}= \ell\hat{y}, \\
z \mathrel{+}= \ell\hat{z}, \nonumber
\end{gather}

\noindent where $\ell$, the sampled travel distance is defined as:

\begin{equation}
\ell = 
	\begin{cases}
	\frac{-\log\xi}{\Sigma_{\mathrm{tot}}} & \text{with ray tracing}, \\
	\frac{-\log\xi}{\Sigma_{\mathrm{maj}}} & \text{with delta-tracking}. \\
	\end{cases}
\end{equation}

If ray tracing is being used, a check is performed between the sampled neutron flight distance and the
distance to the nearest boundary crossing on that trajectory. If the boundary crossing is closer than the
sampled distance, the particle position is instead updated as

\begin{gather}
x \mathrel{+}= s\hat{x} + \varepsilon x_{\mathrm{norm}}, \nonumber\\
y \mathrel{+}= s\hat{y} + \varepsilon y_{\mathrm{norm}}, \\
z \mathrel{+}= s\hat{z} + \varepsilon z_{\mathrm{norm}}, \nonumber
\end{gather}

\noindent where $s$ is the distance to the surface to be crossed. The epsilon term added to the position 
is required
by the use of the OptiX framework; the inclusion of this term ensures that the neutron is sufficiently far
 into the new cell such that it can be detected by OptiX and the corresponding material updated
accordingly. The second term uses vector norms rather than the vector directions in Equation \ref{eq:move}
to move particles away from the surface in a perpendicular manner rather than potentially along a surface
boundary (which could cause issues with OptiX detection). In this case of a particle crossing a boundary
with ray tracing, the particle is stopped at this updated location and the flight path distance must be
resampled using the total macrscopic cross section of the new material.

Although delta-tracking circumvents the above resampling in the case of surface crossings, there is a 
potential need to resample at each collision site. The collision site is accepted as an actual reaction 
with probability

\begin{equation}
P_{\mathrm{col}}(E) = \frac{\Sigma_{\mathrm{tot,col}}(E)}{\Sigma_{\mathrm{maj}}(E)} 
\end{equation}

\noindent and rejected as a ``virtual" collision otherwise. It is important to note that this rejection
sampling must use the total macrscopic cross section of the material in which the collision is being 
considered and not the material in which the neutron was originally located. If the collision is 
considered virtual, a new path length is sampled with the majorant cross section again, and the transport
process continues.

It should be noted that calculating the majorant cross section in WARP is done na{\"i}vely:

\begin{equation}
\label{eq:majnaive}
\Sigma_{\mathrm{maj}}(E)
= \max\{\Sigma_{\mathrm{tot,1}}(E), \Sigma_{\mathrm{tot,2}}(E), \ldots, \Sigma_{\mathrm{tot,M}}(E)\}\:,
\end{equation}

\noindent where $M$ is the number of materials in the entire system. One potential area of development 
would be to
calculate the majorant cross section along the neutron's trajectory (as specified in Equation 
\ref{eq:maj}) rather than this ``global" majorant cross section.

With either tracking method, once the neutron has been determined to undergo an actual collision the 
reaction it will experience is chosen using the microscopic cross sections of the material in which 
the particle is located. Following this, the reaction is processed, and transport continues as above.

\section{Performance potential of delta-tracking in WARP}

Because the OptiX trace must be done twice in each instance of the delta-tracking process and the origina 
ray tracing method only necessitates one trace, it may be that the current implementation of 
delta-tracking may not result be faster than standard ray tracing for simple geometry and 
material configurations. It has been seen in various other Monte Carlo neutron
transport codes that the bulk of the runtime is involved with ray tracing, and WARP is not too different
from other codes in that regard.
This issue could potentially be circumvented if a different framework were to be used rather than OptiX,
but this would require a sizable overhaul of the code and was not considered for this
project. Further, the OptiX framework is optimized for NVIDIA GPUs and at this time it is unclear what
a new, better choice might be. 

However, it is expected that, like in other codes, delta-tracking may be more efficient in WARP when
considering certain geometry configurations. Like Serpent, WARP is intended for calculations in nuclear
reactor analysis. Many reactor configurations consist of lattices composed of many geometry and material
boundaries, causing simulations that use ray tracing to run slowly because 
neutron flights stop at each boundary crossing. Thus, the time saved in not stopping neutron
flights at boundary crossings may compensate for having to call an additional OptiX trace in the 
delta-tracking method implemented in WARP.

